\section{Introduction}
Multidimensional evaluation of abilities has a positive correlation with the performance of upstream tasks. This hypothesis predicates on the assumption that a given test upstream task $\gamma$ is a construction of either orthogonal or related abilities. In this sense, the relative performance of an upstream task can be represented as a vector of abilities $\vec{\gamma} = (\theta_1, \theta_2, \ldots, \theta_n)$ whose magnitude is a measure of the success probability of the model when performing general day-to-day tasks.

\textcolor{red}{Sang: It's often helpful to outline 2-3 main contributions of the analysis in the introduction. For example: 
In summary, our contributions are:
\begin{itemize}
    \item We conduct a multi-dimensional factor analysis to show that there is substantial overlap between abilities tested across datasets.
    \item 2nd contributions
    \item 3rd contributions
\end{itemize}
}